\documentclass[11pt,a4paper]{report}
\usepackage[utf8]{inputenc}
\usepackage[french]{babel}
\usepackage[T1]{fontenc}
\usepackage{amsmath}
\usepackage{amsfonts}
\usepackage{amssymb}
\usepackage{graphicx}
	%table des matières, url cliquables
\usepackage{hyperref}

\usepackage{newcent} %police d'écriture (facultatif)
\usepackage[left=2cm,right=2cm,top=2cm,bottom=2cm]{geometry} % marges

\usepackage{textpos} % positionnement des logos

	% sous-figures
\usepackage{subcaption}

%%% personnalisation aspect des parties (\part, \section, etc...)
\usepackage{titlesec}
% personnalisation chapitre
\titleformat
{\chapter}
[display]
{\centering\normalfont\large\scshape\bfseries}
{\rule[3pt]{0.15\linewidth}{3pt}\quad\chaptertitlename~\thechapter\quad \rule[3pt]{0.15\linewidth}{3pt}}
{2\baselineskip}
{\rule{\linewidth}{0.5pt}\break\Huge}
[\vspace{-0.5\baselineskip}\rule{\linewidth}{0.5pt}\vspace{0.65\baselineskip}]

% personnalisation sections
\titleformat
{\section}
[block]
{\Large\scshape\bfseries}
{\thesection}
{\baselineskip}
{}
[\hrule\vspace{2pt}\hrule\vspace{0.8\baselineskip}]

% personnalisation sous-section
\titleformat
{\subsection}
[block]
{\normalfont\large\bfseries}
{\thesubsection}
{\baselineskip}
{}
[\hrule\vspace{0.75\baselineskip}]

% personnalisation sous-sous-section
\titleformat
{\subsubsection}
[block]
{\itshape\normalsize\bfseries}
{\normalfont\bfseries \thesubsubsection}
{\baselineskip}
{}
[\vspace{0.5\baselineskip}]

%%% personnalisation table des matières
\usepackage[]{titletoc} %pour personnaliser la table des matières

% personnalisation chapitre
\titlecontents
{chapter}
[100pt]
{\addvspace{2pc}}
{\scshape\bfseries\large\contentslabel[Chapitre~\thecontentslabel]{100pt}}
{}
{\hrule\vspace{2pt}\hrule}
[]

% personnalisation section
\titlecontents
{section}
[25pt]
{\addvspace{3pt}}
{\normalfont\scshape\contentslabel{20pt}}
{}
{\normalsize\dotfill\bfseries\thecontentspage}
[]

% personnalisation sous-section
\titlecontents
{subsection}
[75pt]
{}
{\small\contentslabel{30pt}}
{}
{\normalsize\dotfill\small\thecontentspage}
[]

% personnalisation sous-sous-section
\titlecontents
{subsubsection}
[125pt]
{}
{\normalfont\footnotesize\contentslabel{30pt}\itshape}
{}
{\normalsize\dotfill\footnotesize\itshape\thecontentspage}
[]

\newcommand{\HRule}{\rule{\linewidth}{0.5mm}} %faire des traits sur toute la largeur de la page

\begin{document}
\begin{titlepage}

	\begin{textblock*}{6cm}(-10pt,0pt)
	\includegraphics[scale=0.7]{logo_tps.png}
	\end{textblock*}

	\begin{textblock*}{6cm}(375pt,-5pt)
	\includegraphics[scale=0.06]{logo_arts.png}
	\end{textblock*}


  \begin{sffamily}
    \begin{center}

      \vfill
      \textsc{\LARGE Valorisation de l'engagement étudiant}\\[0.5cm]

      \textsc{\Large Association Robot Télécom Strasbourg}\\[1cm]

      % Title
      \HRule\\[0.4cm]
      {\huge \bfseries Rapport intermédiaire\\[0.4cm]}
      \HRule\\[1cm]

      \textsc{\Large Jonathan Plasse}\\[1cm]

      \textsc{\large Vendredi 15 Mars}



      \vfill

      % Bottom of the page
      {\large Année universitaire 2018 -- 2019}

    \end{center}
  \end{sffamily}
\end{titlepage}

\renewcommand{\thepage}{\roman{page}}

\tableofcontents %table des matières

% \listoffigures %table des figures

% \listoftables %table des tableaux

\chapter*{Introduction}
  \renewcommand{\thepage}{\arabic{page}}
  \setcounter{page}{1}

  Dès le début de cette année je me suis investi dans l'association robot dans le but de l'améliorer et la promouvoir.
  Pour cela je me suis investi autant dans les projets robotique de l'ARTS que dans l'administration.
  Mon but étant de mettre en place des systèmes qui permettront la transmission de connaissance et de savoir pour les générations futures de l'association.

  Pour cela, je travaille sur un code de contr\^ole d'une base roulante compréhensible et modulaire afin qu'il soit réutilisable dans les années à venir.
  Je suis aussi en train de réaliser des utilitaires pour pouvoir régler les paramètres des robots facilement. Par exemple, un utilitaire pour régler facilement le PID d'un moteur.

  De plus je m'investis aussi dans l'administration de l'ARTS en tant que responsable coupe et membre du bureau. Je t\^ache à améliorer et faciliter la gestion de l'association.

\chapter{Réalisation de projets robotiques}


  \section{Base roulante}
    \subsection{Pourquoi}
      Une base roulante est ce qui permet au robot de se déplacer. Le but de ce projet de fournir une méthode pour en concevoir et programmer une. Ma partie sur ce projet est la programmation.

    \subsection{Recherches}
      N'ayant aucune connaissance au préalable, j'ai effectuer des recherches sur les façon d'asservir un moteur à courant continu car c'est autour de ceci que l'on peut construire un code de déplacement. J'ai d'ailleurs chercher des projets similaire au mien.
      J'ai couplé la recherche avec l'implémentation de solutions trouvés. Ce qui permet de vérifier si l'on est sur la bonne voie.
      Quelques équipes participant à la coupe de France de robotique on mis à disposition des explications sur le fonctionnement de leur code de déplacement. Comme Cubot, ClubElek, RCVA…
      Au fur et à mesure de mes recherche j'ai établi une liste de fonction à implémenter pour le code de déplacement de la base roulante.

    \subsection{Fonctions à implémenter}
      \subsubsection{Asservissement des moteurs}
        Comme il est expliquer précédemment, l'asservissement des moteurs est la base du code de déplacement. Ici, j'ai choisi d'asservir les moteurs en vitesse.

        Pour réaliser cette asservissement, il faut pouvoir mesurer la vitesse des moteurs. Ainsi, j'utilise les codeurs à quadrature de phase des moteurs.
        Un codeur à quadrature de phase génère deux signaux carré déphasé l'un par rapport à l'autre. On peut ainsi déduire la position en regardant quand l'un des deux signals change.
        %image de quadrature de phase
        Pour récuper la position d'un codeur à quadrature de phase j'utilise la bibliothèque de Encoder de Paul Stoffregen qui utilise les interruptions pour calculer la position d'un codeur. Cette bibliothèque peut utiliser 0, 1 ou 2 interruptions pour un codeur. 2 interruption utilise une interruption pour chaque signal, c'est la solution la plus fiable. 1 interruption n'en utilise qu'une pour un seul signal ce qui est moins fiable mais c'est négligeable dans notre cas car l'arduino est assez rapide. 0 interruption est une solution à éviter dans la mesure du possible.

        Pour ce projet, j'ai choisi d'utiliser la solution avec une interruption par codeur à cause d'une contrainte technique : l'arduino n'a que deux interruptions externes.

        Pour l'asservissement, j'ai utilisé en premier la bibliothèque PID de Brett Beauregard. Mais j'ai ensuite implémenter un PID.
      \subsubsection{Génération des consignes de vitesse}
        J'utilise un asservissement en position au dessus de l'asservissement en vitesse. Un problème est apparu quand j'ai implémenter cette solution car le robot dérape à cause d'une accélération trop forte. Ainsi j'ai implémenter un profil trapésoïdal de vitesse.

        Pour l'asservissement en position il faut aussi connaître la position actuelle du robot. C'est pourquoi j'ai implémenter une odométrie
      \subsubsection{Génération de trajectoire}
        Les déplacements du robot consiste à se déplacer en ligne droite ou à tourner sur lui même. Ainsi, l'odométrie du robot est plus fiable. La génération de trajectoir a pour rôle de générer les commandes avancer en ligne droite ou tourner sur lui même en fonction de la destination et des obstacles fixes (murs) ou mobile (robots).

        Je n'ai pas encore réaliser son implémentation.

  \section{Interface Arduino / PC}
    \subsection{Pourquoi}
      L'interface a pour but d'interragir avec l'arduino depuis un ordinateur, et ainsi pouvoir contr\^oler l'arduino, obtenir des données et déboguer.

    \subsection{Implémentation}
      J'ai choisi d'utiliser Python comme langage de programmation pour le programme sur l'ordinateur car c'est multiplatforme et plus facile à prendre en main.
      Pour le transfert de donnée, j'ai choisi d'envoyer des variables, tableaux ou structure en binaire.

  \section{Utilitaire de réglage de PID}
    \subsection{Pourquoi}
      L'utilitaire de réglage de PID à pour but de d'obtenir facilement les coefficients du PID pour un moteur à courant continu.

    \subsection{Principe}
      Tout d'abord, je crée un modèle du moteur de la forme \(\frac{K\tau}{(s+\tau)}\). Je néglige la constante de temps électrique du système qui est négligeable devant la constante de temps mécanique.

      Pour obtenir la constante de temps mécanique \(\tau \) et le gain du système \(K\), je mesure la réponse indiciel du moteur plusieurs fois pour faire une moyenne. À partir de cette réponse indiciel moyenne, je fais une régression. J'obtiens ainsi \(K\) et \(\tau \).

      Une fois ces constantes obtenus, j'utilise l'outil \texttt{pidTuner}, et obtient ainsi les coefficients du PID en fonction du temps de réponse et l'amortissement.

    \subsection{Futures améliorations}
      J'aimerai ne pas utiliser matlab dans cet utilitaire car j'aimerai partager cet utilitaire or matlab n'est pas gratuit.
      De plus il serait plus pratique de n'avoir qu'un script python qui s'occupe de tout.

  \section{Site Internet}
    \subsection{Pourquoi}
      Besoin de refaire le site web à cause de problème d'hébergement.

    \subsection{Choix}
      \subsubsection{Hébergement}
        J'ai choisi d'utiliser \emph{GitHub Pages} pour héberger le site internet car c'est gratuit facile à maintenir en groupe et nous utilisons déjà GitHub.

      \subsubsection{Technologie}
        Pour générer le site, j'ai choisi \emph{Jekyll} un générateur de site statique. Je l'ai choisi car il est compatible avec github et simble à utilisé.

      \subsubsection{Nom de domaine}
        Le nom de domaine du site internet était initialement :

        \emph{associationrobottelecomstrasbourg.github.io}, c'est l'adresse par défaut que \emph{GitHub} fournit.

        Nous avons donc choisi d'utilisé un nom de domaine de notre choix : \emph{robot-ps.com}

      \subsubsection{Langue}
        Le site internet est en anglais car nous ne voulons pas limiter notre contenu au communauté francophone.

  \section{Future projet}
    Voici les différents futures projets prévu.

    \subsection{Formation sur ESP8266}
      L'ESP8266 est un microcontrolleur qui a un module Wifi intégré et serait utile pour contr\^oller à distance un robot.

    \subsection{Formation sur ROS}
      ROS est très utilisé dans le domaine de la robotique. Apporté à l'association les connaissances pour le prendre en main me semble important.

\chapter{Gestion de l'association}
  En ce début de cette année scolaire 2018/2019, Augustin Bielefeld et moi même avons établis différents sujets de réflexions afin d'améliorer le fonctionnement de l'association.

  \section{Outils de gestion}
    L'ARTS utilisait auparavant deux clouds, un pour les programmes des projets de l'association et l'autre pour tout ce qui concernait l'administration.
    Le seul outils de communication entre les membres de l'association était les mails.
    J'ai ainsi choisi différents service sur Internet pour complémenter ou supprimer les anciens outils de gestion.

    \subsection{Adresse mail}
      L'adresse de l'association club.robot.tephystras@gmail.com n'étant pas claire et explicite. J'ai créer la nouvelle adresse robot.telecom.strasbourg@gmail.com.
      De plus, j'ai utiliser cette nouvelle adresse pour créer un compte sur tout les outils utiliser et chacun d'eux sont administrateur.

    \subsection{Slack}
      \subsubsection{Qu'est ce que Slack}
        Slack est un outil de communication très adapter pour collaborer. Il comporte des chaînes accessible à tout les membres et d'autres privées pour discuter à une personne uniquement. Une chaîne est un fil de discussion où chacun peut poster des messages. Chaque chaîne à un thème, ce qui permet d'organiser les discussions et de facilement retrouver une information.

      \subsubsection{Pourquoi l'utiliser}
        J'ai choisi d'utiliser Slack car il permet aux membres de l'association d'échanger en dehors des séances du samedi après-midi et ainsi cela permet que les différents membres de l'association se tienne informer et motive à travailler en dehors des séances. De plus, la possibilité de séparer les discussions en plusieurs chaînes permet de savoir directement quel est le contexte du message.

      \subsubsection{Mise en place de Slack}
        Pendant une séance de l'ARTS, j'ai expliquer que j'avais mis en place un Slack pour l'ARTS et expliquer pourquoi j'ai mis en place cette outils.
        J'ai invité tout les membres de l'association sur le slack grâce à la liste des adresses mails.

      \subsubsection{Adoption de l'outils}
        Pendant les premières semaines après l'installation du Slack, seul les membres du bureau utilisait le Slack. Pour faire face à ce problème, j'ai rapeller pendant plusieurs séances que l'existence du Slack et peu à peu les membres ce sont mis à interragir.
        Actuellement, le Slack est plut\^ot actif.

    \subsection{Trello}
      \subsubsection{Qu'est ce que Trello}
        Trello est un outil de gestion qui permet de faire des listes. Une liste est constitués de plusieurs cartes (éléments). Une carte peut contenir une description, des checlists, une date d'échéance, des étiquettes, un ou plusieurs responsables et un espace commentaire.

      \subsubsection{Pourquoi l'utiliser}
        J'ai choisi d'utiliser Trello car il permet vraiment de mettre en évidence de manière claire et précise les différents éléments à faire, à venir…

    \subsection{GitHub}
      \subsubsection{Qu'est ce que GitHub}
        \emph{GitHub} est un outil en ligne qui couplé avec git permet de faire du versionnage, sauvegarder et partager ses projets.

      \subsubsection{Pourquoi l'utiliser}
        L'ARTS a beaucoup de projet et produit ainsi beaucoup de code et fichier qui ont besoin d'être ligne pour le partage et 

  \section{Déroulement d'un Conseil d'Administration}
    \subsection{Avant le CA}
    \subsection{Pendant le CA}
    \subsection{Après le CA}

  % \section{Mon poste de responsable coupe}
  %   Voici ce que j'ai fait en tant que responsable coupe depuis le début de l'année.
  %   \subsection{Coupe de France de robotique}
  %     Comme chaque année l'ARTS participe à la coupe de France de robotique
  %     \subsubsection{Inscription sur Poolzor}
  %       changement d'adresse mail
  %       et remplissage des différentes information
  %     \subsubsection{Paiement}
  %       envoie du chèque et surveiller que l'étape du paiement soit validé.
  %   \subsection{Coupe de Belgique}
  %     \subsubsection{Mise en attente}
  %     \subsubsection{Acceptation}
  %     \subsubsection{Estimation des coûts}
  %       Trajet location voiture, carburant, péage et hébergement
  %     \subsubsection{Constitution de l'équipe}
  %     \subsubsection{Paiement}
  %   \subsection{Suvi de l'avancement des projets}
  %
  % \section{Transmission des connaissance}
  %   \subsection{Création de processus sur le Trello}
  %   \subsection{Faire un carnet de passation}
  %   \subsection{Tirer parti de l'engagement étudiant}

\chapter*{Conclusion}

\end{document}
