\documentclass[12pt,a4paper]{report}
\usepackage[utf8]{inputenc}
\usepackage[french]{babel}
\usepackage[T1]{fontenc}
\usepackage{amsmath}
\usepackage{amsfonts}
\usepackage{amssymb}
\usepackage{graphicx}
\usepackage[left=2cm,right=2cm,top=2cm,bottom=2cm]{geometry}

\title{Rapport Intermédiaire}
\author{Jonathan Plasse}
\date{Lundi 4 Mars}

\begin{document}
\maketitle

\chapter*{Introduction}
  Dès le début de cette année je me suis investi dans l'association robot dans le but de l'améliorer et la promouvoir.
  Pour cela je me suis investi autant dans les projets robotique de l'ARTS que dans l'administration.
  Mon but étant de mettre en place des systèmes qui perdure dans le temps et qui facilitera les générations futures de l'association.

  Pour cela, je travaille sur un code de contr\^ole d'une base roulante compréhensible et modulaire afin qu'il soit réutilisable dans les années à venir.
  Je suis aussi en train de réaliser des utilitaires pour pouvoir régler les paramètres des robots facilement. Je suis en train de développer un utilitaire pour régler facilement le PID d'un moteur.

  De plus je m'investis aussi dans l'administration de l'ARTS en tant que membre du bureau et t\^ache à améliorer et faciliter la gestion de l'association.

\chapter{Réalisation de projet}
  \section{Base roulante}
    \subsection{Premier prototype}
    \subsection{Refonte du code}
    \subsection{Création d'une méthode de réglage de PID}
  \section{Site Internet}
    \subsection{Choix de technologie}
    \subsection{Formation}
    \subsection{Réalisation}

\chapter{Gestion de l'association}
  En ce début de cette année scolaire 2018/2019, Augustin Bielefeld et moi même avons établis différents sujets de réflexions afin d'améliorer le fonctionnement de l'association.

  \section{Recrutement}


  \section{Outils de gestion}
    L'ARTS utilisait auparavant deux clouds, un pour les programmes des projets de l'association et l'autre pour tout ce qui concernait l'administration.
    Le seul outils de communication entre les membres de l'association était les mails.
    J'ai ainsi choisi différents service sur Internet pour complémenter ou supprimer les anciens outils de gestion.

    \subsection{Adresse mail}
      L'adresse de l'association club.robot.tephystras@gmail.com n'étant pas claire et explicite. J'ai créer la nouvelle adresse robot.telecom.strasbourg@gmail.com.
      De plus, j'ai utiliser cette nouvelle adresse pour créer un compte sur tout les outils utiliser et chacun d'eux sont administrateur.

    \subsection{Slack}
      \subsubsection{Qu'est ce que Slack}
        Slack est un outil de communication très adapter pour collaborer. Il dispose de chaînes accessible à tout les membres et d'autres privées pour discuter à une personne uniquement. Une chaîne est un fil de discussion où chacun peut poster des messages. Chaque chaîne à un thème, ce qui permet d'organiser les discussions et de facilement retrouver une information.

      \subsubsection{Pourquoi l'utiliser}
        J'ai choisi d'utiliser Slack car il permet aux membres de l'association d'échanger en dehors des séances du samedi après-midi et ainsi cela permet que les différents membres de l'association se tienne informer et motive à travailler en dehors des séances. De plus, la possibilité de séparer les discussions en plusieurs chaînes permet de savoir directement quel est le contexte du message.

      \subsubsection{Mise en place de Slack}
        Pendant une séance de l'ARTS, j'ai expliquer que j'avais mis en place un Slack pour l'ARTS et expliquer pourquoi j'ai mis en place cette outils.
        J'ai invité tout les membres de l'association sur le slack grâce à la liste des adresses mails.

      \subsubsection{Adoption de l'outils}
        Pendant les premières semaines après l'installation du Slack, seul les membres du bureau utilisait le Slack. Pour faire face à ce problème, j'ai rapeller pendant plusieurs séances que l'existence du Slack et peu à peu les membres ce sont mis à interragir.
        Actuellement, le Slack est plut\^ot actif.

    \subsection{Trello}
      \subsubsection{Qu'est ce que Trello}
        Trello est un outil de gestion qui permet de faire des listes. Une liste est constitués de plusieurs cartes (éléments). Une carte peut contenir une description, des checlists, une date d'échéance, des étiquettes, un ou plusieurs responsables et un espace commentaire.

      \subsubsection{Pourquoi l'utiliser}
        J'ai choisi d'utiliser Trello

    \subsection{GitHub}

  \section{Déroulement d'un Conseil d'Administration}

  \section{Transmission des connaissance}

\chapter*{Conclusion}

\end{document}
